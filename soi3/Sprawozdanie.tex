\documentclass{mwrep}

% Polskie znaki
\usepackage{polski}
\usepackage[utf8]{inputenc}
\usepackage[T1]{fontenc}
\usepackage{lmodern}
\usepackage{indentfirst}

% Strona tytułowa
\usepackage{pgfplots}
\usepackage{siunitx}
\usepackage{paracol}

% Pływające obrazki
\usepackage{float}
\usepackage{svg}
\usepackage{graphicx}

% table of contents refs
\usepackage{hyperref}
\usepackage{cleveref}
\usepackage{booktabs}
\usepackage{listings}


\SendSettingsToPgf
\title{\bf Zarządzanie pamięcią \\ Raport \vskip 0.1cm}
\author{Jakub Sikora}
\date{\today}
\pgfplotsset{compat=1.15}	
\begin{document}

\makeatletter
\renewcommand{\maketitle}{\begin{titlepage}
		\begin{center}{
				\LARGE {\bf Politechnika Warszawska}}\\
			\vspace{0.4cm}
			{\LARGE {\bf Wydział Elektroniki i Technik Informacyjnych}}\\
			\vspace{5cm}
			{\bf \LARGE \mbox{Systemy operacyjne} \vskip 0.1cm}
		\end{center}
		\vspace{0.1cm}

		\begin{center}
			{\bf \LARGE \@title}
		\end{center}

		\vspace{10cm}
		\begin{paracol}{2}
			\addtocontents{toc}{\protect\setcounter{tocdepth}{1}}
			\subsection*{Zdający:}
			\bf{ \Large{ \noindent\@author \par}}
			\addtocontents{toc}{\protect\setcounter{tocdepth}{2}}

			\switchcolumn \addtocontents{toc}{\protect\setcounter{tocdepth}{1}}
			\subsection*{Prowadzący:}
			\bf{\Large{\noindent mgr. inż. Aleksander \\ Pruszkowski}}
			\addtocontents{toc}{\protect\setcounter{tocdepth}{2}}

		\end{paracol}
		\vspace*{\stretch{6}}
		\begin{center}
			\bf{\large{Warszawa, \@date\vskip 0.1cm}}
		\end{center}
	\end{titlepage}
}
\makeatother
\maketitle

\tableofcontents

\chapter{Treść zadania}
Zgodnie z wytycznymi podanymi na stronie "Ćwiczenie 3
Synchronizacja procesów z wykorzystaniem semaforów" , napisz usługę „chat”
dającą użytkownikom możliwość komunikacji między sobą (w obrębie jednej maszyny).
Załóż, że niektórzy wśród użytkowników mają podwyższone prawa i ich wiadomości
„wskakują” na początek kolejki rozsyłania wiadomości do pozostałych użytkowników.
Załóż, że wszyscy użytkownicy komunikują się poprzez system w wyłącznie jednym temacie.
Przemyśl metodę testowania powstałego systemu, w szczególności zwróć uwagę na
pokazanie w działaniu równoczesnego „chatowania” obu grup użytkowników – w tym
udowodnienia, że system faktycznie obsługuje wiadomości uwzględniając priorytety.
Załóż, że wytworzone przez Ciebie programy testowe będą działały automatycznie
generując testowe fikcyjne wiadomości a jeden z nich będzie te
wiadomości pokazywał na konsoli.

\chapter{Realizacja rozwiązania}
Usługę "chatowania" zrealizowałem za pomocą zestawu semaforów systemowych,
chroniących jednoczesnego dostępu do dwóch buforów cyklicznych zaalokowanych
w pamięci współdzielonej.

\section{Bufory cykliczne}
Wiadomości przekazywane w ramach usługi "chatu" przechowywane były w buforach cyklicznych.
Każdy z buforów alokował pamięć na pewną ilość elementów dalej oznaczaną \texttt{MAX} oraz na dwa wskaźniki: odczytu i zapisu.
Proces zapisujący nową wiadomość zapisuje ją pod adresem wskazywanym przez wskaźnik zapisu a następnie
przesuwa go o jedną pozycję do przodu. Jeśli w momencie dodawania nowej wiadomości, wskaźnik zapisu
znajdzie się na ostatniej pozycji bufora(indeks \texttt{MAX-1}), proces przesuwa go na pozycję pierwszą o indeksie zero.
Analogicznie, proces odczytujący wiadomość z bufora pobiera ją z miejsca wskazywanego przez wskaźnik
odczytu a następnie przesuwa go o jedną pozycję. Mechanizmy synchronizacji powinny zapewnić że tylko jeden
proces może aktualnie pracować na buforze cyklicznym oraz muszą zabezpieczyć przed wyprzedzeniem wskaźnika zapisu
przez wskaźnik odczytu.\\ 
\indent W pojedyncznym buforze cyklicznym, nie jest możliwe poprawne obsłużenie mechanizmu priorytetowego pobierania
wiadomości specjalnych, dlatego też w przedstawionym rozwiązaniu zaimplementowałem dwa bufory z czego pierwszy
obsługuje tylko wiadomości wysłane przez procesy uprzywilejowane a drugi tylko wiadomości zwykłe.\\
\indent Bufory cykliczne zostały zaimplementowane w pamięci współdzielonej. Inicjalizację pamięci
wykonałem za pomocą funkcji \texttt{shmget} a wskaźnik na zaalokowaną pamięc uzyskałem za pomocą procedury \texttt{shmat}.


\section{Semafory}
Aby umożliwić synchronizację procesów wysyłających (producentów) oraz odbierającego (konsumenta), skorzystałem
z zestawu semaforów oferowanych przez system operacyjny \texttt{Linux}. Do poprawnej obsługi problemu, należało
skorzystać z pięciu semaforów. Para semaforów (w przedstawionym rozwiązaniu oznaczonych numerami $0$ i $1$) chroni przed
niepoprawnym dostępem do bufora cyklicznego. Semafor o numerze $0$ inicjowany wartością \texttt{MAX} wskazuje ile jest wolnych
slotów w buforze. Wartość ta jest dekrementowana przez producentów oraz dekrementowana przez konsumenta. Cechą semaforów jest fakt,
że próba dekrementacji semafora, którego wartośc wynosi $0$, spowoduje zawieszenie procesu do momentu zwolnienia się miejsca w buforze (czyli
do momentu gdy konsument zinkrementuje wartość tej zmiennej poprzez pobranie elementu z bufora, zwalniając przy tym slot na następną wiadomość).
Semafor o numerze $1$ działa analogicznie do poprzedniego. Inicjowany wartością $0$, zlicza ile elementów jest gotowych do pobrania przez
konsumenta. Semafor ten zabezpiecza przed odczytem pustego bufora. Konsument próbując odczytać nową wiadomość z pustego bufora,
zostanie zawieszony do momentu gdy któryś z producentów włoży do bufora nowy element. Analogicznie działają semafory o numerach
$2$ i $3$. Pilnują one przed niepoprawnym dostępem do bufora trzymającego dane priorytetowe. \\

\indent Ostatni semafor oznaczony numerem $4$, zabezpiecza oba bufory przed jednoczesnym dostępem wielu procesów.
Inicjowany wartością $1$, pilnuje aby tylko jeden proces mogł edytować dowolny z buforów. \\

\indent Zbiór semaforów uzyskałem za pomocą polecenia \texttt{semget} a operacje inkrementacji i dekrementacji wykonywałem
za pomocą procedury \texttt{semop}.


\section{Przykład producenta}
\label{producent}
\indent Przykładowa operacja wstawiania nowej wiadomości do bufora wygląda w następujący sposób:

\begin{verbatim}
    decrease(semid, 0);
    decrease(semid, 4);
    buf[indexZ] = arg;
    indexZ = (indexZ+1) % MAX;
    increase(semid, 1);
    increase(semid, 4);
\end{verbatim}

W pierwszej kolejności, proces wstawiający sprawdza czy jest miejsce w buforze
poprzez dekrementację wartości semafora $0$. Producent przechodzącąc do następnego
polecenia, niejako rezerwuje sobie wolny slot w buforze. W linijce drugiej, proces wstrzymuje się
do momentu aż będzie mógł wejść do sekcji krytycznej i będzie miał dostęp do buforów na wyłączność.
Linijki 3 i 4 wykonują operację zapisu do bufora cyklicznego i inkrementację wskaźnika zapisu.
W ostatnich dwóch linijkach proces zwiększa ilość elementów gotowych w buforze oraz zwiększa semafor $4$,
informując inne procesy że wyszedł z sekcji krytycznej.


\section{Przykład konsumenta}
\label{konsument}
\indent Przykładowa operacja konsumenta jest nieco bardziej złożona, ponieważ musi brać pod
uwagę istnienie dwóch buforów:

\begin{verbatim}
    while(1){
        decrease(semid, 4); 
        if(semctl(semid, 3, GETVAL) != 0){
            decrease(semid, 3);
            printf("%5d\n", buf[indexOpri])
            indexOpri = (indexOpri+1)%MAX;
            increase(semid, 2);
            increase(semid, 4);
            continue;
        }
        else{
            if(semctl(semid, 1, GETVAL) != 0){
                decrease(semid, 1);
                printf("%5d\n", buf[indexO])
                indexO = (indexO+1)%MAX;
                increase(semid, 0);
                increase(semid, 4);
            }
            else {
                increase(semid, 4);
            }
        }
    }

\end{verbatim}

\indent Program pojedynczego konsumenta z zadaną częstotliwością skanuje sekcję krytyczną
w poszukiwaniu nowych wiadomości. Konsument wchodzi do sekcji krytycznej poprzez dekrementację semafora $4$.
Wiedząc że tylko jeden proces może się znajdować w sekcji krytycznej oraz że to konsument się w niej znajduje,
może on sprawdzić wartości semaforów bez obaw że ich wartość może się zmienić bez jego wiedzy.
W pierwszej kolejności proces sprawdza czy są jakieś wiadomości priorytetowe. Jeśli nie ma żadnych,
proces sprawdza czy są jakieś wiadomości zwyczajne. Jeżeli w danym buforze są wiadomości i konsument zdecydował
się wyjąć z niego wiadomość, proces dekrementuje semafor który przechowuje informację o ilości zajętych slotów.
Zużycie wiadomości następuje poprzez wypisanie treści oraz przesunięcie wskaźnika odczytu. Ostatecznie
inkrementowane są wartości semaforów oznaczających ilość pustych slotów w buforze oraz semafor $4$
implementujący wzajemne wykluczanie sygnalizując wyjście z sekcji krytycznej.


\chapter{Testowanie rozwiązania}

\section{Programy testowe}
Aby przeprowadzić testy przedstawionego rozwiązania przygotowałem kilka programów w języku C.
Każdy z programów sprawdza przy uruchomieniu czy pod danym stałym kluczem zainicjalizowane zostały już
mechanizmy pamięci współdzielonej. Jeśli nie, same dokonują inicjalizacji.

\subsection{Program czytający - \texttt{rx}}
Program czytający \texttt{rx} pełni w przedstawionym problemie rolę konsumenta. Imituje on
konsolę na której wypisywane są wiadomości przesłane przez producentów. Program co okres $1s$, sprawdza
czy są jakieś wiadomości do odczytania zgodnie z mechanizmem przedstawionym w sekcji \ref{konsument}.

\subsection{Programy wysyłające - \texttt{tx}, \texttt{spamtx}, \texttt{bursttx}}
\label{tx}
Programy wysyłające w przedstawionym problemie pełni rolę producenta. Wszystkie przedstawione programy
działają zgodnie z zasadą przedstawioną w sekcji \ref{producent}. Różnią się tylko częstotliwością działania.
Program \texttt{tx} wysyła łącznie dziesięć wiadomości w odstępach $2$ sekund. Program \texttt{spamtx}
również wysyła dziesięć wiadomości, jednak wysyła je bez odstępów czasowych. Ostatni program \texttt{bursttx} wysyła
wiadomości w sposób podobny do \texttt{spamtx}, jednak ilość wysłanych wiadomości wynosi $500$. Jest to ilość wystarczająca
do przepełnienia bufora oraz prezentacji działania rozwiązania w takich okolicznościach. Programy te jako argument wywołania pobuierją numer,
który będzie wysyłany jako wiadomość.

\subsection{Programy priorytetowe - \texttt{px}, \texttt{spampx}, \texttt{burstpx}}
Programy priorytetowe działają na tej samej zasadzie na jakiej działają zwykłe programy wysyłające opisane
w podsekcji \ref{tx}. Różnią się tylko destynacją wiadomości. Wysyłają one wyprodukowane
wiadomości do bufora priorytetowego \texttt{buf\_pri}.

\section{Scenariusze testowe}
W ramach zadania przygotowałem kilka scenariuszy testowych prezentujących poprawność
zaimplementowanego rozwiązania.

\subsection{Odczyt z pustego bufora}
W pierwszym teście dokonałem próby odczytu z pustego bufora. W tym celu uruchomiłem tylko program czytający \texttt{rx}.

\begin{verbatim}
    sikora@dell:~/Documents/Dev/Soi/SOI-laboratorium/soi3$ ./rx
    Brak wiadomości w buforze.
    Brak wiadomości w buforze.
    Brak wiadomości w buforze.
    Brak wiadomości w buforze.
\end{verbatim}

W każdej iteracji, program wchodził do sekcji krytycznej i sprawdzał czy w obu buforach są jakieś dane
do wyświetlenia. Za każdym razem program żadnych danych nie znajdywał, co sygnalizował stosownym komunikatem.
Brak wiadomości w buforach nie powodował zawieszenia programu czytającego.

\subsection{Zachowanie kolejności wiadomości}
W kolejnym teście uruchomiłem trzy procesy nadawcze \texttt{tx} oraz jeden proces odbiorczy. W ramach tego testu
sprawdziłem czy elementy o równym priorytecie wychodzą z bufora w takiej samej kolejności w jakiej do niego weszły.\\

\noindent Test wykonałem za pomocą skryptu \texttt{test1.sh}.
\begin{verbatim}
    ./tx 200&
    sleep 0.1
    ./tx 300&
    sleep 0.1
    ./tx 400&
\end{verbatim}

Wynik testu znajduje się poniżej.

\begin{paracol}{2}
\begin{verbatim}
./rx
2018-12- 5  0:28:43   200
2018-12- 5  0:28:44   300
2018-12- 5  0:28:45   400
2018-12- 5  0:28:46   200
2018-12- 5  0:28:47   300
2018-12- 5  0:28:48   400
2018-12- 5  0:28:49   200
2018-12- 5  0:28:50   300
2018-12- 5  0:28:51   400
2018-12- 5  0:28:52   200
2018-12- 5  0:28:53   300
2018-12- 5  0:28:54   400
2018-12- 5  0:28:55   200
2018-12- 5  0:28:56   300
2018-12- 5  0:28:57   400
2018-12- 5  0:28:58   200
2018-12- 5  0:28:59   300
2018-12- 5  0:29: 0   400
2018-12- 5  0:29: 1   200
2018-12- 5  0:29: 2   300
2018-12- 5  0:29: 3   400
2018-12- 5  0:29: 4   200
2018-12- 5  0:29: 5   300
2018-12- 5  0:29: 6   400
2018-12- 5  0:29: 7   200
2018-12- 5  0:29: 8   300
2018-12- 5  0:29: 9   400
2018-12- 5  0:29:10   200
2018-12- 5  0:29:11   300
2018-12- 5  0:29:12   400
\end{verbatim}
\switchcolumn
\begin{verbatim}
./test1.sh
2018-12- 5  0:28:43: 0. Wysłano: 200
2018-12- 5  0:28:43: 0. Wysłano: 300
2018-12- 5  0:28:43: 0. Wysłano: 400
2018-12- 5  0:28:45: 1. Wysłano: 200
2018-12- 5  0:28:45: 1. Wysłano: 300
2018-12- 5  0:28:45: 1. Wysłano: 400
2018-12- 5  0:28:47: 2. Wysłano: 200
2018-12- 5  0:28:47: 2. Wysłano: 300
2018-12- 5  0:28:47: 2. Wysłano: 400
2018-12- 5  0:28:49: 3. Wysłano: 200
2018-12- 5  0:28:49: 3. Wysłano: 300
2018-12- 5  0:28:49: 3. Wysłano: 400
2018-12- 5  0:28:51: 4. Wysłano: 200
2018-12- 5  0:28:51: 4. Wysłano: 300
2018-12- 5  0:28:51: 4. Wysłano: 400
2018-12- 5  0:28:53: 5. Wysłano: 200
2018-12- 5  0:28:53: 5. Wysłano: 300
2018-12- 5  0:28:53: 5. Wysłano: 400
2018-12- 5  0:28:55: 6. Wysłano: 200
2018-12- 5  0:28:55: 6. Wysłano: 300
2018-12- 5  0:28:55: 6. Wysłano: 400
2018-12- 5  0:28:57: 7. Wysłano: 200
2018-12- 5  0:28:57: 7. Wysłano: 300
2018-12- 5  0:28:57: 7. Wysłano: 400
2018-12- 5  0:28:59: 8. Wysłano: 200
2018-12- 5  0:28:59: 8. Wysłano: 300
2018-12- 5  0:28:59: 8. Wysłano: 400
2018-12- 5  0:29: 1: 9. Wysłano: 200
2018-12- 5  0:29: 1: 9. Wysłano: 300
2018-12- 5  0:29: 1: 9. Wysłano: 400
\end{verbatim}
\end{paracol}

Listing powyżej jednoznacznie udowadnia że zaimplementowane rozwiązanie
poprawnie zachowuje kolejność wiadomości przychodzących.

\newpage
\subsection{Priorytet wiadomości}
Kolejny test miał na celu sprawdzić czy przedstawione rozwiązanie poprawnie interpretuje wiadomości
z priorytetem i obsługuje je jako pierwsze. W tym celu napisałem kolejny skrypt o nazwie \texttt{test2.sh}.

\begin{verbatim}
    ./tx 200&
    ./tx 300&
    ./tx 400&
    sleep 1.5s
    ./px 9999&
\end{verbatim}

W tym teście, procesy zwykłe wysyłały wiadomość co dwie sekundy a proces priorytetowy wysyłał wiadomości
co trzy sekundy. Proces konsumencki skanował wiadomości co jedną sekundę.
Wynik testu znajduje się poniżej.

\begin{paracol}{2}
\begin{verbatim}
./rx
2018-12- 5  0:42:25   200
2018-12- 5  0:42:26   400
2018-12- 5  0:42:27  9999
2018-12- 5  0:42:28   300
2018-12- 5  0:42:29   200
2018-12- 5  0:42:30  9999
2018-12- 5  0:42:31   400
2018-12- 5  0:42:32   300
2018-12- 5  0:42:33  9999
2018-12- 5  0:42:34   200
2018-12- 5  0:42:35   400
2018-12- 5  0:42:36  9999
2018-12- 5  0:42:37   300
2018-12- 5  0:42:38   200
2018-12- 5  0:42:39  9999
2018-12- 5  0:42:40   400
2018-12- 5  0:42:41   300
2018-12- 5  0:42:42  9999
2018-12- 5  0:42:43   200
2018-12- 5  0:42:44   400
2018-12- 5  0:42:45  9999
2018-12- 5  0:42:46   300
Brak wiadomości w buforze.
2018-12- 5  0:42:48  9999
Brak wiadomości w buforze.
Brak wiadomości w buforze.
2018-12- 5  0:42:51  9999
Brak wiadomości w buforze.
Brak wiadomości w buforze.
2018-12- 5  0:42:54  9999
\end{verbatim}
	\switchcolumn
	\begin{verbatim}
./test2.sh
2018-12- 5  0:42:25: 0. Wysłano: 200
2018-12- 5  0:42:25: 0. Wysłano: 400
2018-12- 5  0:42:25: 0. Wysłano: 300
2018-12- 5  0:42:27: 0. Wysłano: 9999
2018-12- 5  0:42:27: 1. Wysłano: 200
2018-12- 5  0:42:27: 1. Wysłano: 400
2018-12- 5  0:42:27: 1. Wysłano: 300
2018-12- 5  0:42:29: 2. Wysłano: 200
2018-12- 5  0:42:29: 2. Wysłano: 400
2018-12- 5  0:42:29: 2. Wysłano: 300
2018-12- 5  0:42:30: 1. Wysłano: 9999
2018-12- 5  0:42:31: 3. Wysłano: 200
2018-12- 5  0:42:31: 3. Wysłano: 400
2018-12- 5  0:42:31: 3. Wysłano: 300
2018-12- 5  0:42:33: 2. Wysłano: 9999
2018-12- 5  0:42:33: 4. Wysłano: 200
2018-12- 5  0:42:33: 4. Wysłano: 400
2018-12- 5  0:42:33: 4. Wysłano: 300
2018-12- 5  0:42:36: 3. Wysłano: 9999
2018-12- 5  0:42:39: 4. Wysłano: 9999
2018-12- 5  0:42:42: 5. Wysłano: 9999
2018-12- 5  0:42:45: 6. Wysłano: 9999
2018-12- 5  0:42:48: 7. Wysłano: 9999
2018-12- 5  0:42:51: 8. Wysłano: 9999
2018-12- 5  0:42:54: 9. Wysłano: 9999
\end{verbatim}
\end{paracol}

Na podstawie znajomości częstotliwości z jakimi iterują dane procesy oraz znaczników czasowych, można jednoznacznie
stwierdzić że zaimplementowane przeze mnie rozwiąznie pomyślnie obsługuje wiadomości priorytetowe. Nawet przy
sporej ilości oczekujących wiadomości zwykłych, w pierwszej kolejności i tak są brane pod uwagę wiadomości priorytetowe.
\newpage

\subsection{Przepełnienie bufora}
W tym teście sprawdziłem jak rozwiązanie radzi sobie z przepełnieniem bufora na wiadomości. W związku z tym
uruchomiłem tylko proces wysyłający równocześnie $500$ wiadomości zwykłych. Jako że maksymalna pojemność bufora 
została przeze mnie ustalona na $400$, proces powinien zawiesić się przed wysłaniem $401$ wiadomości. Przy poprawnej implementacji
rozwiązania, proces powinien oczekiwać na konsumenta wiadomości i dopiero wtedy zacząć wysyłać nowe dane, gdy z proces czytający
zacznie opróżniać bufor. \\

\indent Test zrealizowałem za pomocą programu \texttt{bursttx} opisanego w sekcji \ref{tx}.

\begin{verbatim}
./bursttx 200
2018-12- 5  1: 1:40: 0. Wysłano: 200
2018-12- 5  1: 1:40: 1. Wysłano: 200
2018-12- 5  1: 1:40: 2. Wysłano: 200
2018-12- 5  1: 1:40: 3. Wysłano: 200
\end{verbatim}
\vdots
\begin{verbatim}
2018-12- 5  1: 1:40: 395. Wysłano: 200
2018-12- 5  1: 1:40: 396. Wysłano: 200
2018-12- 5  1: 1:40: 397. Wysłano: 200
2018-12- 5  1: 1:40: 398. Wysłano: 200
2018-12- 5  1: 1:40: 399. Wysłano: 200
\end{verbatim}

Zgodnie z oczekiwaniami, proces zatrzymał się po wysłaniu $400$ wiadomości. Zaimplementowane skutecznie
zapobiegło przepełnieniu bufora.

\subsection{Zagłodzenie procesów}
Ostatni test miał na celu sprawdzić czy nie występuje zjawisko głodzenia procesów. 
W tym scenariuszu sprawdzałem czy w przypadku w całości zapełnionego bufora, stopniowo opróżnianego przez konsumenta
dostęp do nowopowstałego slotu będzie rodzielany pomiędzy oczekujące procesy. W tym celu uruchomiłem program czytający \texttt{rx} oraz
dwóch nadawców \texttt{tx} z argumentami $200$ i $300$. \\

Poniżej znajduje się końcówka uzyskanych z eksperymentu logów nadawców.

\begin{verbatim}
./test6.sh
\end{verbatim}
\vdots
\begin{verbatim}
2018-12- 5  1: 4:20: 198. Wysłano: 300
2018-12- 5  1: 4:21: 229. Wysłano: 200
2018-12- 5  1: 4:22: 199. Wysłano: 300
2018-12- 5  1: 4:23: 230. Wysłano: 200
2018-12- 5  1: 4:24: 200. Wysłano: 300
2018-12- 5  1: 4:25: 231. Wysłano: 200
\end{verbatim}

Na podstawie testu można stwierdzić że każdy z procesów jest równomiernie dopuszczany do sekcji krytycznej.
Róznice w indeksach wiadomości wynikają z nierównego startu obu procesów.

\section{Podsumowanie}
Przedstawiony powyżej zestaw testów jednoznacznie potwierdza poprawność przedstawionego rozwiązania oraz że 
skutecznie rozwiązuje najpoważniejsze problemy synchronizacyjne w przypadku problemu konsumenta i wielu producentów o
różnych priorytetach.


\end{document}